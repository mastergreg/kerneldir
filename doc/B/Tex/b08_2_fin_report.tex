\documentclass[a4paper,11pt]{article} \usepackage{anysize}
\marginsize{1cm}{1cm}{1cm}{1cm}
%\textwidth 6.0in \textheight = 664pt
\usepackage{xltxtra}
\usepackage{xunicode}
\usepackage{graphicx}
\usepackage{color}
\usepackage[usenames,dvipsnames]{xcolor}
\usepackage{xgreek}
\usepackage{fancyvrb}
\usepackage{minted}
\usepackage{listings}
\usepackage{enumitem} \usepackage{framed} \usepackage{relsize}
\usepackage{float} \setmainfont[Mapping=TeX-text]{FreeSerif}
\begin{document}

\renewcommand{\theenumi}{\roman{enumi}}
\begin{titlepage}
    \begin{center}
        \begin{figure}[t] 
            \includegraphics[scale=0.7]{title/ntua_logo}
        \end{figure}
    \begin{LARGE}\textbf{ΕΘΝΙΚΟ ΜΕΤΣΟΒΙΟ ΠΟΛΥΤΕΧΝΕΙΟ\\}\end{LARGE}
        \vspace{2cm}
        \begin{Large}
            ΣΧΟΛΗ ΗΜ\&ΜΥ\\
            Εργαστήριο Λειτουργικών Συστημάτων
            %1\textsuperscript{η} Άσκηση\\
            Ακ. έτος 2011-2012\\
        \end{Large}
        \vspace{5cm}
        \Large Τμήμα Β, Ομάδα 8\textsuperscript{η}\\
        \vspace{1cm}
        \begin{tabular}{l r r}
            \Large{Γερακάρης Βασίλης}&
            \large{Α.Μ.: 03108092}&
            \small{<vgerak@gmail.com>}\\
            \Large{Λύρας Γρηγόρης}&
            \large{Α.Μ.: 03109687}&
            \small{<gregliras@gmail.com>}\\
        \end{tabular}\\
        \vspace{5cm}

        \vfill
        \large\today\\
    \end{center}
\end{titlepage}



\section*{sys\_curse: σχεδιασμός υποδομής σημείωσης διεργασιών στον πυρήνα Linux}
\subsection*{σχεδιασμός υποδομής curse}
\begin{itemize}
    \item τι επεμβάσεις στον κώδικα ή τις δομές δεδομένων του πυρήνα προδιαγράψατε και γιατί.

        Θα τροποποιηθεί το task\_struct ώστε να περιέχει την πληροφορία για
        τις κατάρες καθώς και όσες κλήσεις ή κομμάτια του συστήματος θέλουμε
        να γνωρίζουν και να αναλαμβάνουν κάποια δράση ανάλογα με την κατάρα
        που έχει η διεργασία.

    \item πώς εξασφαλίσατε την κληρονομικότητα των κατάρων.

        Πρίν την ολοκλήρωση του fork προσθέσαμε κώδικα ώστε να αντιγράφεται το
        πεδίο των καταρών όταν αυτές είναι κληρονομικές (προεπιλογή).


    \item τι προβλέπει ο σχεδιασμός σας για το ενδεχόμενο ταυτόχρονης πρόσβασης πολλών διεργασιών.

        Ο μηχανισμός θα συγχρονίζεται με χρήση spinlock.

    \item με ποιο τρόπο ένας προγραμματιστής μπορεί να υλοποιήσει μια κατάρα με την υποδομή σας.

        Για να προσθέσει ένας προγραμματιστής μια κατάρα θα πρέπει να κάνει τα
        εξής:
        \begin{itemize}
            \item Να προσθέσει την κατάρα του στο Kconfig του curse\_imp ώστε 
                η κατάρα να είναι διαθέσιμη κατά το configuration του πυρήνα.
            \item Να δημιουργήσει ένα καινούριο ζευγάρι αρχείων c,h για την
                κατάρα όπου θα περιέχονται μία έως τρείς συναρτήσεις
                init, inject και destroy. 
            \item Να ορίσει την κατάρα μέσα στο αρχείο curse\_list.h με το
                όνομά της και τον μοναδικό 64bit κωδικό της επιλογής του καθώς
                και τις προηγούμενες συναρτήσεις της (αν κάποια δεν υπάρχει
                μπορεί να χρησιμοποιηθεί η αντίστοιχη stub συνάρτηση).
            \item Να θέσει το trigger για την κατάρα του όπου εκείνος νομίζει
                μέσα στον πυρήνα με τη χρήση του curse\_trigger.
            \item Αν η κατάρα χρειάζεται private data υπάρχει η
                curse\_create\_alloc και η curse\_free\_alloc για να δεσμεύουν
                μνήμη για ιδιωτικά δεδομένα των κατάρων.
        \end{itemize}


    \item πώς εξασφαλίσατε ότι το ίδιο εκτελέσιμο αρχείο μπορεί να λειτουργήσει σε διαφορετικά συστήματα με διαφορετικές κατάρες.

        Το εκτελέσιμο απλά ζητάει από τον πυρήνα να θέσει στη διεργασία x την
        κατάρα y. Το εκτελέσιμο δε χρειάζεται να γνωρίζει κάτι άλλο για τη
        συγκεκριμένη κατάρα, το εκτελέσιμο θα ζητήσει από τη βιβλιοθήκη να
        καταραστεί μια διεργασία με τη συγκεκριμένη κατάρα καλώντας την
        με το όνομά της. Η βιβλιοθήκη ζητάει από τον πυρήνα τις διαθέσιμες
        κατάρες και ψάχνει σε αυτές για να βρει αυτή που ζήτησε ο χρήστης.
        Έτσι η υλοποίηση κρύβεται από το χρήστη και το ίδιο εκτελέσιμο μπορεί 
        να χρησιμοποιηθεί ανεξάρτητα από το ποιες κατάρες είναι υλοποιημένες.

    \item ποια εκτιμάτε ότι θα είναι η επιβάρυνση των παρεμβάσεών σας στο λειτουργικό σύστημα;

        Η επιβάρυνση στο σύστημα θα είναι ελάχιστη. Κάθε κατάρα καταλαμβάνει
        τον ελάχιστο δυνατό χώρο στη μνήμη. Προς το παρόν τα private data
        αντιγράφονται σε κάθε fork ωστόσο αυτό είναι δυνατό να διορθωθεί
        χρησιμοποιώντας έναν reference counter και υλοποιώντας πολιτική COW αν
        αυτή κριθεί απαραίτητη. 

\end{itemize}

\pagebreak

\subsection*{Παράρτημα: Πηγαίος κώδικας}


\subsubsection*{curse.h}
\lstinputlisting[breaklines=true,tabsize=4, language=C]{files/curse.h}
\subsubsection*{curse.c}
\lstinputlisting[breaklines=true,tabsize=4, language=C]{files/curse.c}

\subsubsection*{curse\_externals.h}
\lstinputlisting[breaklines=true,tabsize=4, language=C]{files/curse_externals.h}
\subsubsection*{curse\_externals.c}
\lstinputlisting[breaklines=true,tabsize=4, language=C]{files/curse_externals.c}
\subsubsection*{curse\_list.h}
\lstinputlisting[breaklines=true,tabsize=4, language=C]{files/curse_list.h}
\subsubsection*{curse\_types.h}
\lstinputlisting[breaklines=true,tabsize=4, language=C]{files/curse_types.h}

\subsubsection*{no\_curse.h}
\lstinputlisting[breaklines=true,tabsize=4, language=C]{files/no_curse.h}
\subsubsection*{no\_curse.c}
\lstinputlisting[breaklines=true,tabsize=4, language=C]{files/no_curse.c}

\subsubsection*{no\_exit.h}
\lstinputlisting[breaklines=true,tabsize=4, language=C]{files/no_exit.h}
\subsubsection*{no\_exit.c}
\lstinputlisting[breaklines=true,tabsize=4, language=C]{files/no_exit.c}

\subsubsection*{no\_fs\_cache.h}
\lstinputlisting[breaklines=true,tabsize=4, language=C]{files/no_fs_cache.h}
\subsubsection*{no\_fs\_cache.c}
\lstinputlisting[breaklines=true,tabsize=4, language=C]{files/no_fs_cache.c}

\subsubsection*{poison.h}
\lstinputlisting[breaklines=true,tabsize=4, language=C]{files/poison.h}
\subsubsection*{poison.c}
\lstinputlisting[breaklines=true,tabsize=4, language=C]{files/poison.c}

\subsubsection*{random\_oops.h}
\lstinputlisting[breaklines=true,tabsize=4, language=C]{files/random_oops.h}
\subsubsection*{random\_oops.c}
\lstinputlisting[breaklines=true,tabsize=4, language=C]{files/random_oops.c}

\subsubsection*{stub\_curse.h}
\lstinputlisting[breaklines=true,tabsize=4, language=C]{files/stub_curse.h}
\subsubsection*{stub\_curse.c}
\lstinputlisting[breaklines=true,tabsize=4, language=C]{files/stub_curse.c}

\subsubsection*{test\_curse.h}
\lstinputlisting[breaklines=true,tabsize=4, language=C]{files/test_curse.h}
\subsubsection*{test\_curse.c}
\lstinputlisting[breaklines=true,tabsize=4, language=C]{files/test_curse.c}

\end{document}

